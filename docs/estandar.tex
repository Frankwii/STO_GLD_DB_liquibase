\documentclass{article}

\usepackage[catalan]{babel}
\usepackage{geometry}
\usepackage{fancyhdr}
\usepackage{graphicx}
\usepackage{listings}
\usepackage{xcolor}
\usepackage{amsthm}
\usepackage{cleveref}

\title{Estàndard de la base de dades de \\l'Observatori de Turisme Sostenible}
\author{Frank William Hammond Espinosa}
\date{Darrera actualització: \today}

\pagestyle{fancy}
\fancyhf{}
\lhead{Estàndard BDD OTS}
\rhead{Darrera actualització: \today}

\setlength{\parindent}{0px}
\setlength{\parskip}{1em}

\theoremstyle{definition}
\newtheorem{definition}{Definició}

\begin{document}
\maketitle
\tableofcontents

A aquest document es detallarà la metodologia a seguir a l'hora de crear i omplir taules a la base de dades d'aquest repositori.

Per qualsevol dubte relacionat amb aquest document, enviau un mail a frank.hammond@fueib.org o avazquez1@conselldemallorca.net.

\section{Com llegir aquest document}\label{section:howtoread}
This is work in progress.

\section{Nomenclatura de taules i columnes}\label{section:nomenclature}
\subsection{Taules}\label{section:nomenclature:tables}
\subsubsection{Semàntica}\label{section:nomenclature:tables:semantics}
La majoria de taules presents a la BDD a dia d'avui estan pensades per ser utilitzades a quadres de comandament (\emph{dashboards}) fets amb PowerBI. Per aquest motiu es segueix una nomenclatura inspirada en la manera habitual de separar la informació en el món de l'analítica de negoci.

Les taules es separen en dos tipus: de \textbf{dimensió} (\emph{dimension}) i de \textbf{fets} (\emph{facts}).

\begin{definition}[Taules de dimensió]\label{def:tables:dimension}
  Les \textbf{taules de dimensió} són les que contenen els detalls dels conceptes individuals que conté la base de dades. Sempre tenen un identificador únic per fila (\emph{primary key} o clau primària) i tota la informació emmagatzemada dels elements que representen.
\end{definition}

A aquestes taules normalment s'inclouen els noms i/o descripcions dels objectes modelitzats a la base de dades, sovint en diferents idiomes (una columna per idioma). També s'anoten les relacions a altres objectes de la base de dades en forma de \emph{foreign keys} o claus forànies. Per exemple, per representar la relació ``cada ciutat pertany a un país'', a la taula de ciutat es guarda una columna amb una clau forània a l'identificador de la taula de país.

Per conveni, a aquesta base de dades les claus primàries sempre estaran formades per una única columna de tipus \verb|INT| (enter). 

\begin{definition}[Taules de fets]\label{def:tables:facts}
  Les \textbf{taules de fets} són les que contenen les relacions ``reals'' que s'han enregistrat a la base de dades. Normalment, aquestes dades que són les que s'empren a nivell últim per construir gràfiques, i no es solen creuar (operacions \verb|JOIN|) entre elles.
\end{definition}

Sempre que sigui possible, les columnes d'aquestes taules han de contenir identificadors numèrics de tots els elements que contenguin, és a dir, claus forànies a alguna taula de dimensió. En cap cas podran contenir descripcions textuals (\emph{strings}) que apareguin a més d'una fila. Veure \cref{section:normalization} per més detalls sobre guardar identificadors numèrics o descripcions textuals i \cref{section:nomenclature:tables:examples} per exemples de taules.

\subsubsection{Notació}\label{section:nomenclature:tables:notation}
Primerament, els noms de les taules han d'anar prefixats pels literals \verb|DIM_| o \verb|FAC_| segons siguin de dimensió o de fets (veure \cref{def:tables:dimension} i \cref{def:tables:facts} per a una explicació detallada dels conceptes). En cas que en un futur fos necessari ampliar la classificació de taules (per exemple, afegir taules de metadades), s'afegiria un prefixe de tres lletres seguit per un guió baix, semblant als definits prèviament.

Després del prefixe, s'inclou el nom de l'objecte que modelitza la taula. Això, evidentment, depèn molt del cas concret, però s'han de seguir una sèrie de directrius generals que s'inclouen a continuació. Tot i així, ha de prevalèixer el sentit comú del programador, ja que és una qüestió a decidir cas per cas.

\begin{enumerate}
  \item Es segueix la convenció informàtica ``\emph{screaming snake case}''. Això vol dir que els noms de les taules han d'estar enterament en majúscules, composts únicament per caràcters alfanumèrics ASCII (nombres i lletres de la A a la Z, sense accents) i barres baixes (\_), i les paraules han d'estar separades per barres baixes en comptes d'espais. En cap cas s'admeten signes de puntuació llevat de la barra baixa. Això inclou espais en blanc, guions (-), barres verticals o inclinades (\verb'|', \verb|\|, \verb|/|), cometes (\verb|'|, \verb|"|), signes de puntuació estàndard (\verb|,|, \verb|.|, \verb|;|, \verb|:|), etcètera.

  \item Els noms han d'estar en anglès sempre que sigui possible. Noms o sigles d'organitzacions internacionals sempre han d'incloure's en anglès també (per exemple, \emph{UN} en comptes d'\emph{ONU}).

  Excepcions notables a aquesta regla són els topònims o noms de demarcacions territorials nacionals o comunitàries, o organismes propis. Aquests es mantindran en la llengua o sigles locals, en castellà si són entitats nacionals i en català si són balears. Per exemple, una hipotètica taula que guardi informació respecte els diferents consells insulars de les balears s'hauria d'anomenar \verb|DIM_CONSELL_INSULAR|; per les diferents comunitats autònomes d'Espanya, una taula \verb|DIM_COMUNIDAD_AUTONOMA| o \verb|DIM_INE_REGION| per una taula de regions de països segons l'Institut Nacional d'Estadísitca (noti's que no es tradueixen les sigles).

  \item Si la taula modelitza algun concepte en mode que cada fila correspongui a un únic objecte d'aquell tipus, el nom ha d'anar en singular (e.g. \verb|DIM_CITY| en comptes de \verb|DIM_CITIES| per a una taula que guardi informació referent a ciutats).

  \item És preferible utilitzar noms sencers a diminitius o sigles - encara que això resulti en noms verbosos - sempre que els diminutius no siguin part del llenguatge quotidià. Per exemple, no s'ha de posar \verb|CA| en comptes de \verb|COMUNIDAD_AUTONOMA|, perquè no és immediat inferir a què es refereixen aquelles sigles. En canvi, sí és recomanable emprar \verb|UN|, \verb|INE| o \verb|FRONTUR| com a part del nom d'una taula perquè són abreujacions ben establertes i que es podrien emprar com a part d'una frase en l'idioma corresponent sence necessitat d'aclarir què signifiquen.
\end{enumerate}

\subsubsection{Casos d'estudi}\label{section:nomenclature:tables:examples}
This is work in progress.

\subsection{Columnes}\label{section:nomenclature:columns}
\subsubsection{Semàntica}\label{section:nomenclature:columns:semantics}
Normalment, la majoria de columnes 

\subsubsection{Notació}\label{section:nomenclature:columns:notation}
This is work in progress.

\subsubsection{Casos d'estudi}\label{section:nomenclature:columns:examples}
This is work in progress.

\section{Normalització}\label{section:normalization}
This is work in progress.
\subsection{Relacions estàndard}\label{section:normalization:standard}
This is work in progress.


\subsection{Relacions jeràrquiques}\label{sectino:normalization:hierarchical}
This is work in progress.

\end{document}
